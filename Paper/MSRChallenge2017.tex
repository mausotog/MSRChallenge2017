
%% bare_conf.tex
%% V1.3
%% 2007/01/11
%% by Michael Shell
%% See:
%% http://www.michaelshell.org/
%% for current contact information.
%%
%% This is a skeleton file demonstrating the use of IEEEtran.cls
%% (requires IEEEtran.cls version 1.7 or later) with an IEEE conference paper.
%%
%% Support sites:
%% http://www.michaelshell.org/tex/ieeetran/
%% http://www.ctan.org/tex-archive/macros/latex/contrib/IEEEtran/
%% and
%% http://www.ieee.org/

%%*************************************************************************
%% Legal Notice:
%% This code is offered as-is without any warranty either expressed or
%% implied; without even the implied warranty of MERCHANTABILITY or
%% FITNESS FOR A PARTICULAR PURPOSE! 
%% User assumes all risk.
%% In no event shall IEEE or any contributor to this code be liable for
%% any damages or losses, including, but not limited to, incidental,
%% consequential, or any other damages, resulting from the use or misuse
%% of any information contained here.
%%
%% All comments are the opinions of their respective authors and are not
%% necessarily endorsed by the IEEE.
%%
%% This work is distributed under the LaTeX Project Public License (LPPL)
%% ( http://www.latex-project.org/ ) version 1.3, and may be freely used,
%% distributed and modified. A copy of the LPPL, version 1.3, is included
%% in the base LaTeX documentation of all distributions of LaTeX released
%% 2003/12/01 or later.
%% Retain all contribution notices and credits.
%% ** Modified files should be clearly indicated as such, including  **
%% ** renaming them and changing author support contact information. **
%%
%% File list of work: IEEEtran.cls, IEEEtran_HOWTO.pdf, bare_adv.tex,
%%                    bare_conf.tex, bare_jrnl.tex, bare_jrnl_compsoc.tex
%%*************************************************************************

% *** Authors should verify (and, if needed, correct) their LaTeX system  ***
% *** with the testflow diagnostic prior to trusting their LaTeX platform ***
% *** with production work. IEEE's font choices can trigger bugs that do  ***
% *** not appear when using other class files.                            ***
% The testflow support page is at:
% http://www.michaelshell.org/tex/testflow/



% Note that the a4paper option is mainly intended so that authors in
% countries using A4 can easily print to A4 and see how their papers will
% look in print - the typesetting of the document will not typically be
% affected with changes in paper size (but the bottom and side margins will).
% Use the testflow package mentioned above to verify correct handling of
% both paper sizes by the user's LaTeX system.
%
% Also note that the "draftcls" or "draftclsnofoot", not "draft", option
% should be used if it is desired that the figures are to be displayed in
% draft mode.
%
\documentclass[10pt, conference]{IEEEtran}
% Add the compsocconf option for Computer Society conferences.
%
% If IEEEtran.cls has not been installed into the LaTeX system files,
% manually specify the path to it like:
% \documentclass[conference]{../sty/IEEEtran}



\usepackage{booktabs}
\usepackage{verbatim}
\newcommand{\ra}[1]{\renewcommand{\arraystretch}{#1}}

\usepackage[usenames,dvipsnames]{color}
\newcommand{\todo}[1]
  {{\scriptsize \textbf{\color{red} {#1}}}}
%\newcommand{\todo}[1] {}
\newcommand{\clg}[1]
  {{\scriptsize \textbf{\color{blue} {CLG says: #1}}}}


% Some very useful LaTeX packages include:
% (uncomment the ones you want to load)


% *** MISC UTILITY PACKAGES ***
%
%\usepackage{ifpdf}
% Heiko Oberdiek's ifpdf.sty is very useful if you need conditional
% compilation based on whether the output is pdf or dvi.
% usage:
% \ifpdf
%   % pdf code
% \else
%   % dvi code
% \fi
% The latest version of ifpdf.sty can be obtained from:
% http://www.ctan.org/tex-archive/macros/latex/contrib/oberdiek/
% Also, note that IEEEtran.cls V1.7 and later provides a builtin
% \ifCLASSINFOpdf conditional that works the same way.
% When switching from latex to pdflatex and vice-versa, the compiler may
% have to be run twice to clear warning/error messages.






% *** CITATION PACKAGES ***
%
\usepackage{cite}
% cite.sty was written by Donald Arseneau
% V1.6 and later of IEEEtran pre-defines the format of the cite.sty package
% \cite{} output to follow that of IEEE. Loading the cite package will
% result in citation numbers being automatically sorted and properly
% "compressed/ranged". e.g., [1], [9], [2], [7], [5], [6] without using
% cite.sty will become [1], [2], [5]--[7], [9] using cite.sty. cite.sty's
% \cite will automatically add leading space, if needed. Use cite.sty's
% noadjust option (cite.sty V3.8 and later) if you want to turn this off.
% cite.sty is already installed on most LaTeX systems. Be sure and use
% version 4.0 (2003-05-27) and later if using hyperref.sty. cite.sty does
% not currently provide for hyperlinked citations.
% The latest version can be obtained at:
% http://www.ctan.org/tex-archive/macros/latex/contrib/cite/
% The documentation is contained in the cite.sty file itself.






% *** GRAPHICS RELATED PACKAGES ***
%
\ifCLASSINFOpdf
  % \usepackage[pdftex]{graphicx}
  % declare the path(s) where your graphic files are
  % \graphicspath{{../pdf/}{../jpeg/}}
  % and their extensions so you won't have to specify these with
  % every instance of \includegraphics
  % \DeclareGraphicsExtensions{.pdf,.jpeg,.png}
\else
  % or other class option (dvipsone, dvipdf, if not using dvips). graphicx
  % will default to the driver specified in the system graphics.cfg if no
  % driver is specified.
  % \usepackage[dvips]{graphicx}
  % declare the path(s) where your graphic files are
  % \graphicspath{{../eps/}}
  % and their extensions so you won't have to specify these with
  % every instance of \includegraphics
  % \DeclareGraphicsExtensions{.eps}
\fi
% graphicx was written by David Carlisle and Sebastian Rahtz. It is
% required if you want graphics, photos, etc. graphicx.sty is already
% installed on most LaTeX systems. The latest version and documentation can
% be obtained at: 
% http://www.ctan.org/tex-archive/macros/latex/required/graphics/
% Another good source of documentation is "Using Imported Graphics in
% LaTeX2e" by Keith Reckdahl which can be found as epslatex.ps or
% epslatex.pdf at: http://www.ctan.org/tex-archive/info/
%
% latex, and pdflatex in dvi mode, support graphics in encapsulated
% postscript (.eps) format. pdflatex in pdf mode supports graphics
% in .pdf, .jpeg, .png and .mps (metapost) formats. Users should ensure
% that all non-photo figures use a vector format (.eps, .pdf, .mps) and
% not a bitmapped formats (.jpeg, .png). IEEE frowns on bitmapped formats
% which can result in "jaggedy"/blurry rendering of lines and letters as
% well as large increases in file sizes.
%
% You can find documentation about the pdfTeX application at:
% http://www.tug.org/applications/pdftex





% *** MATH PACKAGES ***
%
%\usepackage[cmex10]{amsmath}
% A popular package from the American Mathematical Society that provides
% many useful and powerful commands for dealing with mathematics. If using
% it, be sure to load this package with the cmex10 option to ensure that
% only type 1 fonts will utilized at all point sizes. Without this option,
% it is possible that some math symbols, particularly those within
% footnotes, will be rendered in bitmap form which will result in a
% document that can not be IEEE Xplore compliant!
%
% Also, note that the amsmath package sets \interdisplaylinepenalty to 10000
% thus preventing page breaks from occurring within multiline equations. Use:
%\interdisplaylinepenalty=2500
% after loading amsmath to restore such page breaks as IEEEtran.cls normally
% does. amsmath.sty is already installed on most LaTeX systems. The latest
% version and documentation can be obtained at:
% http://www.ctan.org/tex-archive/macros/latex/required/amslatex/math/





% *** SPECIALIZED LIST PACKAGES ***
%
%\usepackage{algorithmic}
% algorithmic.sty was written by Peter Williams and Rogerio Brito.
% This package provides an algorithmic environment fo describing algorithms.
% You can use the algorithmic environment in-text or within a figure
% environment to provide for a floating algorithm. Do NOT use the algorithm
% floating environment provided by algorithm.sty (by the same authors) or
% algorithm2e.sty (by Christophe Fiorio) as IEEE does not use dedicated
% algorithm float types and packages that provide these will not provide
% correct IEEE style captions. The latest version and documentation of
% algorithmic.sty can be obtained at:
% http://www.ctan.org/tex-archive/macros/latex/contrib/algorithms/
% There is also a support site at:
% http://algorithms.berlios.de/index.html
% Also of interest may be the (relatively newer and more customizable)
% algorithmicx.sty package by Szasz Janos:
% http://www.ctan.org/tex-archive/macros/latex/contrib/algorithmicx/




% *** ALIGNMENT PACKAGES ***
%
%\usepackage{array}
% Frank Mittelbach's and David Carlisle's array.sty patches and improves
% the standard LaTeX2e array and tabular environments to provide better
% appearance and additional user controls. As the default LaTeX2e table
% generation code is lacking to the point of almost being broken with
% respect to the quality of the end results, all users are strongly
% advised to use an enhanced (at the very least that provided by array.sty)
% set of table tools. array.sty is already installed on most systems. The
% latest version and documentation can be obtained at:
% http://www.ctan.org/tex-archive/macros/latex/required/tools/


%\usepackage{mdwmath}
%\usepackage{mdwtab}
% Also highly recommended is Mark Wooding's extremely powerful MDW tools,
% especially mdwmath.sty and mdwtab.sty which are used to format equations
% and tables, respectively. The MDWtools set is already installed on most
% LaTeX systems. The lastest version and documentation is available at:
% http://www.ctan.org/tex-archive/macros/latex/contrib/mdwtools/


% IEEEtran contains the IEEEeqnarray family of commands that can be used to
% generate multiline equations as well as matrices, tables, etc., of high
% quality.


%\usepackage{eqparbox}
% Also of notable interest is Scott Pakin's eqparbox package for creating
% (automatically sized) equal width boxes - aka "natural width parboxes".
% Available at:
% http://www.ctan.org/tex-archive/macros/latex/contrib/eqparbox/





% *** SUBFIGURE PACKAGES ***
%\usepackage[tight,footnotesize]{subfigure}
% subfigure.sty was written by Steven Douglas Cochran. This package makes it
% easy to put subfigures in your figures. e.g., "Figure 1a and 1b". For IEEE
% work, it is a good idea to load it with the tight package option to reduce
% the amount of white space around the subfigures. subfigure.sty is already
% installed on most LaTeX systems. The latest version and documentation can
% be obtained at:
% http://www.ctan.org/tex-archive/obsolete/macros/latex/contrib/subfigure/
% subfigure.sty has been superceeded by subfig.sty.



%\usepackage[caption=false]{caption}
%\usepackage[font=footnotesize]{subfig}
% subfig.sty, also written by Steven Douglas Cochran, is the modern
% replacement for subfigure.sty. However, subfig.sty requires and
% automatically loads Axel Sommerfeldt's caption.sty which will override
% IEEEtran.cls handling of captions and this will result in nonIEEE style
% figure/table captions. To prevent this problem, be sure and preload
% caption.sty with its "caption=false" package option. This is will preserve
% IEEEtran.cls handing of captions. Version 1.3 (2005/06/28) and later 
% (recommended due to many improvements over 1.2) of subfig.sty supports
% the caption=false option directly:
%\usepackage[caption=false,font=footnotesize]{subfig}
%
% The latest version and documentation can be obtained at:
% http://www.ctan.org/tex-archive/macros/latex/contrib/subfig/
% The latest version and documentation of caption.sty can be obtained at:
% http://www.ctan.org/tex-archive/macros/latex/contrib/caption/




% *** FLOAT PACKAGES ***
%
%\usepackage{fixltx2e}
% fixltx2e, the successor to the earlier fix2col.sty, was written by
% Frank Mittelbach and David Carlisle. This package corrects a few problems
% in the LaTeX2e kernel, the most notable of which is that in current
% LaTeX2e releases, the ordering of single and double column floats is not
% guaranteed to be preserved. Thus, an unpatched LaTeX2e can allow a
% single column figure to be placed prior to an earlier double column
% figure. The latest version and documentation can be found at:
% http://www.ctan.org/tex-archive/macros/latex/base/



%\usepackage{stfloats}
% stfloats.sty was written by Sigitas Tolusis. This package gives LaTeX2e
% the ability to do double column floats at the bottom of the page as well
% as the top. (e.g., "\begin{figure*}[!b]" is not normally possible in
% LaTeX2e). It also provides a command:
%\fnbelowfloat
% to enable the placement of footnotes below bottom floats (the standard
% LaTeX2e kernel puts them above bottom floats). This is an invasive package
% which rewrites many portions of the LaTeX2e float routines. It may not work
% with other packages that modify the LaTeX2e float routines. The latest
% version and documentation can be obtained at:
% http://www.ctan.org/tex-archive/macros/latex/contrib/sttools/
% Documentation is contained in the stfloats.sty comments as well as in the
% presfull.pdf file. Do not use the stfloats baselinefloat ability as IEEE
% does not allow \baselineskip to stretch. Authors submitting work to the
% IEEE should note that IEEE rarely uses double column equations and
% that authors should try to avoid such use. Do not be tempted to use the
% cuted.sty or midfloat.sty packages (also by Sigitas Tolusis) as IEEE does
% not format its papers in such ways.





% *** PDF, URL AND HYPERLINK PACKAGES ***
%
%\usepackage{url}
% url.sty was written by Donald Arseneau. It provides better support for
% handling and breaking URLs. url.sty is already installed on most LaTeX
% systems. The latest version can be obtained at:
% http://www.ctan.org/tex-archive/macros/latex/contrib/misc/
% Read the url.sty source comments for usage information. Basically,
% \url{my_url_here}.





% *** Do not adjust lengths that control margins, column widths, etc. ***
% *** Do not use packages that alter fonts (such as pslatex).         ***
% There should be no need to do such things with IEEEtran.cls V1.6 and later.
% (Unless specifically asked to do so by the journal or conference you plan
% to submit to, of course. )


% correct bad hyphenation here
\hyphenation{op-tical net-works semi-conduc-tor}


\begin{document}
%
% paper title
% can use linebreaks \\ within to get better formatting as desired

\title{Analyzing the impact of social attributes on commit integration success}
%Or some better name, Zack can probably think of a better one, he is good with names
%ZC- Haha thanks.  I'm starting to lean to very descriptive titles now.  They
%are nice for searching.


% author names and affiliations
% use a multiple column layout for up to two different
% affiliations

\author{\IEEEauthorblockN{Authors hidden for the purposes of double blind review}
\IEEEauthorblockA{\\
\\
\\}

%\author{\IEEEauthorblockN{Authors Name/s per 1st Affiliation (Author)}
%\IEEEauthorblockA{line 1 (of Affiliation): dept. name of organization\\
%line 2: name of organization, acronyms acceptable\\
%line 3: City, Country\\
%line 4: Email: name@xyz.com}
%\and
%\IEEEauthorblockN{Authors Name/s per 2nd Affiliation (Author)}
%\IEEEauthorblockA{line 1 (of Affiliation): dept. name of organization\\
%line 2: name of organization, acronyms acceptable\\
%line 3: City, Country\\
%line 4: Email: name@xyz.com}
}

% conference papers do not typically use \thanks and this command
% is locked out in conference mode. If really needed, such as for
% the acknowledgment of grants, issue a \IEEEoverridecommandlockouts
% after \documentclass

% for over three affiliations, or if they all won't fit within the width
% of the page, use this alternative format:
% 
%\author{\IEEEauthorblockN{Michael Shell\IEEEauthorrefmark{1},
%Homer Simpson\IEEEauthorrefmark{2},
%James Kirk\IEEEauthorrefmark{3}, 
%Montgomery Scott\IEEEauthorrefmark{3} and
%Eldon Tyrell\IEEEauthorrefmark{4}}
%\IEEEauthorblockA{\IEEEauthorrefmark{1}School of Electrical and Computer Engineering\\
%Georgia Institute of Technology,
%Atlanta, Georgia 30332--0250\\ Email: see http://www.michaelshell.org/contact.html}
%\IEEEauthorblockA{\IEEEauthorrefmark{2}Twentieth Century Fox, Springfield, USA\\
%Email: homer@thesimpsons.com}
%\IEEEauthorblockA{\IEEEauthorrefmark{3}Starfleet Academy, San Francisco, California 96678-2391\\
%Telephone: (800) 555--1212, Fax: (888) 555--1212}
%\IEEEauthorblockA{\IEEEauthorrefmark{4}Tyrell Inc., 123 Replicant Street, Los Angeles, California 90210--4321}}




% use for special paper notices
%\IEEEspecialpapernotice{(Invited Paper)}




% make the title area
\maketitle


\begin{abstract}
As the software development community continues to make it easier for everyone
to contribute, the number of commits and pull requests keep increasing. However, this exciting growth
renders it more difficult to only accept quality contributions.
Recent research has found 
that both technical and social factors predict
the success of project contributions on GitHub.  
We take this question a step further, focusing on predicting continuous
integration build success based on technical and
social factors involved in a commit.
Specifically, we investigated if social factors (such as being a
core member of the development team, having a large number of followers,
or contributing a large number of commits) improve predictions of build
success.  We found that social factors cause a noticeable increase
in predictive power (12\%), core team members are more likely to pass the build
tests (10\%), and users with 1000 or more follower are more likely to
pass the build tests (10\%).
\end{abstract}

\begin{IEEEkeywords}
Social Attributes; Travis CI; GitHub; Social Networks; Predicting Integration 
Success; Social Coding
\end{IEEEkeywords}


% For peer review papers, you can put extra information on the cover
% page as needed:
% \ifCLASSOPTIONpeerreview
% \begin{center} \bfseries EDICS Category: 3-BBND \end{center}
% \fi
%
% For peerreview papers, this IEEEtran command inserts a page break and
% creates the second title. It will be ignored for other modes.
\IEEEpeerreviewmaketitle



\section{Introduction}
For managers of an open source project, it is important to ensure that the
project consists of high quality contributions.  As social coding sites attract
more users and projects become more popular, they receive ever more
community contributions.  As the number of contributions to a project increase,
the project must dedicate more time to vetting these contributions. 
%
Over the years, the open source
community has developed various techniques to vet project contributions, such as pull
requests which must be approved by a core team member~\cite{gousios14}. 
These methods can become a bottleneck, and thus it is beneficial to prioritize certain
contributions.  This prioritization can be achieved with predictors of a
contribution's quality.

The decision to include code in
a project has been described as a meritocracy, where only the code is a factor
in the decision~\cite{Scacchi07}.
While one would think that only the complexity of the commit is important, past
studies have shown that the developer making the commit is also an important
indicator when investigating contribution quality~\cite{tsay14icse, tsay14fse}.
Thus, both technical factors, which describe the complexity of the commit, and
social factors, which describe the developer's experience, are important when
investigating contribution quality.

This previous work has focused on the predicted acceptance of \emph{pull requests}. 
By contrast,  we are interested in analyzing the quality 
of a contribution in the context of continuous integration.  While pull requests
quality is determined through manual inspection, continuous integration quality
is automatically determined through test cases, and thus approach the problem of
quality determination differently.
We investigate the factors involved in continuous integration build success.
In this study, we analyze a large set of technical and social attributes 
and how these can be used to predict a contributions quality, measured by the
success of build integration jobs.
We performed this analysis on a large corpus of open source projects
which incorporated continuous integration in their development practice. 
When a contribution to these projects passes integration tests, 
it suggests that the project developers
might have confidence in the changes. 
Thus, we use the commit's build integration status as a proxy for its quality.

In particular, we analyzed contributions in the 
context of continuous integration, using the technical factors and one social
factor found in the challenge
corpus~\cite{msr17challenge}, and the social factors found on a developer's
GitHub profile.  We used the technical factors to estimate the complexity of the
commit, and the social factors to estimate the developer's experience.
Overall, our study investigates how well different technical and social factors predict
contribution quality. This information can help core team members prioritize project
contributions, and will also provide a further understand of quality enforcement
in the software development process.

We find that social factors are an important aspect of build success but
these factors alone do not fully predict if a contribution will be successfully 
integrated into the project. We find
that core team members are more likely to achieve build success than non-core
members, and that developers with more followers are more likely to submit commits that pass the
continuous integration build.

\vspace{1ex}
\noindent\textbf{Related work:}
Similar studies have shown the relationship between both social and technical attributes attributes and 
different measures of success: pull request acceptance~\cite{gousios14, 
gousios15,tsay14icse,tsay14fse}, pull request evaluation latency~\cite{Yu15}.
Tsay et al.~\cite{tsay14icse} 
show that the social connection between the developer initiating a
pull request and the project integrators is a key factor in predicting 
acceptance of a pull 
request. Unlike this previous study, in our study, we look at a broader aspect of 
social aspects, and focus on predicting the success in Travis
CI jobs.Gousios et al.~\cite{gousios14} found that the change location of a commit was
an important indicator in the acceptance rate of those commits by the project.
Also, the inclusion of test cases with a commit has been shown to increase the
commit's acceptance rate~\cite{gousios15}.
Bettenburgh et 
al.~\cite{bettenburgh10} discuss the impact of social interactions with 
post-release defects, Vasilescu et al.~\cite{vasilescu14} explore how different 
project characteristics (programming language, project age, etc.) affect commit 
success when integrated into the automatic build process. Yu et al.~\cite{yu16} 
analyze the impact of CI usage in GitHub and its correlation with software 
quality. By contrast, we focus on the relationship between independent social 
attributes and continuous integration success. 

\section{Research Design}

We present the approach taken to guide this study by
following the Goal-Question-Metric (GQM) methodology described by Basili et 
al.~\cite{Basili84}.

\vspace{1ex}
\noindent\textbf{Goal.}
The goal of this study is to identify and understand the relationship 
between the social factors that describe the
developer's experience (stars, followers, etc.) and technical properties that
describe the commit's complexity
(files added, files modified, etc.)
when predicting the likelihood a developer's commit
to be considered successfully integrated in the project. We define successful
integration as passing the continuous integration build tests in the project.

\vspace{1ex}
\noindent\textbf{Question.}
%Previous studies have identified a relationship between technical and social 
%factors and pull request acceptance. 
We are interested in understanding whether
social factors, used as a proxy for experience, can be used to predict
a different success measure: commit integration success, therefore giving
developers and integrators another tool to be able to identify the quality
of a contribution performed by a particular developer.
%
Our main research question, upon which we can build by investigating particular attributes, is:

\textbf{Do social attributes, gathered 
from the GitHub profiles of developers,
allow for more accurate predictions than only using technical data?}

\begin{comment}
we have created a dataset with technical attributes to be able to analyze the 
predictive power of these attributes and its relationship with commit integration 
success. This way, we are able to create a model
We are afterwards interested in knowing if adding social attributes to this 
predictability model will increase the probability of the model asserting 
correctly future instances by knowing not only technical data about the commit 
being pushed but also social attributes of the developer pushing the changes.
\end{comment}

\vspace{1ex}
\noindent\textbf{Metric.} We measure quality software contributions based on the build status of the
commit. We used a decision tree classifier~\cite{Quinlan86} to identify
important factors in predicting build success;  decision tress are well
established  and interpretable. 
We first build a
decision tree classifier using only technical factors, which highlights
the degree to which commit complexity predicts build integration
success.  We then added the social factors to identify the degree to which they
add predictive power to a the initial technical model. 
We use 10-fold cross validation across our evaluation. 

\section{Building the Dataset}
\label{approach}

We built a data set of
attributes consisting of both technical and social factors:

\vspace{1ex}
\noindent\textbf{Technical Factors} We collected technical factors from the 
MSR Challenge data set~\cite{msr17challenge}.
We extracted only the attributes in the data set that proxy a commit's
complexity, such as commit size. 
The technical factors used are:
\begin{itemize}
%\item tr\_original\_commit
%\item gh\_build\_started\_at 
\item Source churn, changed lines of code.
\item Files added by this commit.
\item Files modified by this commit.
\item Tests added by this commit.
\item Tests deleted by this commit.
\item Total source files in project at the time of the commit.
\item Source lines of source code in the project at
	the time of the commit.
\item Travis status, or status of the Travis build integration.
\end{itemize}

The first five factors represent the changes caused by the commit.  The next two
factors cover the size of the project.  These factors represent the complexity
of the change, and the complexity of the project incorporating the change. 

\vspace{1ex}
\noindent\textbf{Social Factors.} We created a web scraper to extract the social data
from developer's profiles.
We used the commit hash and the project name in the original data
set provided by the MSR Challenge~\cite{msr17challenge} to locate thecommit web page
of the commit, leveraging \textit{github.com}'s URL standard for projects and
commit hashes.
%  using the syntax
% below. 
% \newline \newline
% \textit{https://github.com/\textbf{NameOfProject}/commit/\textbf{CommitHash}}
% \newline \newline
%
We scraped the commit author's
\textit{username} from each commit page, and then
collected the following from the user's public profile:
%(\textit{https://github.com/\textbf{UserName}}). 
%

\begin{itemize}
\item Repositories: number of repositories the developer can access, including those
the developer owns, contributes
to, and for which the developer has organizational membership access.%
\footnote{https://developer.github.com/v3/repos/}
\item Stars: a count of the projects the developer has stared. Starring a project
to shows approval of the repository and creates  a bookmark for later 
access.\footnote{https://help.github.com/articles/about-stars/}
\item Followers: number of developers that follow the author. 
\item Following: number of developers that the author follows 
\item Contributions in the last year: number of contributions made by the author in the last year.
\end{itemize}


These social factors are all factors that are tied to a GitHub user's profile
and that are generally correlated with user time and engagement with the site.
Thus, we use them as a proxy for
experience.
We also added one item from the MSR Challenge data set, indiating
whether the commit's developer is a core member  
of the associated project.

\vspace{1ex}
\noindent\textbf{Comparing two data sets.} We compiled the gathered data into 
two data sets: one with only the technical
factors and the build status, and another with both technical and social
factors and the build status. We performed this split to analyze how well technical factors alone,
and then both technical and social
factors predict build integration status.

Our data set~\cite{msr17challenge} consists
of commits that date as far back as 2011. The social data we 
gather is recent, gathered in January 2017. To mitigate the threat of
using future results to predict previous events,
we removed only include data set commits from the years 2014--2016.  Our intuition is
that the social attributes are less likely to have substantively changed more recently.

Since we are interested in the success of the job status of each contribution, we 
converted all Travis build statuses to two possible values:
\emph{successful} (passed in the original provided data set) and \emph{unsuccessful}
(errored, failed, canceled
and started in the original data set). We also condensed down all passing and
failing builds in a commit to a single pass or fail for each build status.  If
the commit contained both passing and failing build statuses, we treat it as a failed build. 

74.43\% of the builds in the reduced dataset were successful, and thus a naive
classifer that always guesses ``successful'' would succeed 74.43\% of the time.
To penalize misclassification for
both classes equally, we created a second balanced dataset by randomly 
sub-sampling the successful builds until the two build status were
equal, a process known as downsampling.  We show results on both the 
downsampled and non-downsampled data sets.

\section{Evaluation}
\begin{table*}[ht]
\centering
\begin{tabular}{l||r|r||r|r}
  & \multicolumn{2}{|c||}{Non-Downsampled}  & \multicolumn{2}{|c}{Downsampled} \\
\hline
  & Technical & Technical and Social & Technical & Technical and Social \\
\hline
\textbf{Correctly Classified Instances}&  77.01\% & 81.10\%  & 64.10\% & 76.09\%\\

\textbf{Incorrectly Classified Instances}&  22.99\% & 18.90\% & 35.90\% & 23.91\%\\

\textbf{Kappa statistic}& 0.21 &  0.42 & 0.28  & 0.52 \\

\textbf{Mean absolute error}& 0.34 &  0.28 & 0.43 & 0.32\\

%\textbf{Relative absolute error}& 88.64\% & 72.42\%  & 85.35\% & 63.07\%\\
\hline
\textbf{Increase when adding social attributes}& \multicolumn{2}{|c||}{4.09\%}  & \multicolumn{2}{|c}{11.99\%}\\

\hline
\end{tabular}
\\
\center
	\caption{\label{resultsTable}Comparison of predictive power by running 10-fold
    cross validation of two decision trees with a non-downsampled dataset (2nd and 3rd column) and downsampled (4th and 5th column). 
	%The higher the percentages represent the higher number of instances we are able to correctly predict. \label{resultsTable}
	%Zack: the above sentence isn't true for the second row.
	}   
\end{table*}
To address our main research question: ``Do social attributes, gathered 
from the GitHub profiles of developers,
allow for more accurate predictions than only using technical data?", we 
created a decision tree classifier~\cite{Quinlan86}.
Decision trees are a predictive model which uses binary decisions on the
attributes to predict a final class.  The predicted class is represented as the
leaves, while the attribute decisions used in the prediction cause the
different branches.
In this study, we used the Weka J48 decision tree~\cite{Weka,Quinlan1993}
with default settings. The decision trees were evaluated using 10-fold cross 
validation with the data in each data set.

In Section~\ref{sec:successClassification} we discuss the build classification
success rate for the technical and social factors.  In Section~\ref{sec:core},
we compare the build success rate of core project members to non-core project
members.  In Section~\ref{sec:followerSuccess}, we compare the relationship
between a developer's followers and the build integration success.

\subsection{Classification with and without social factors}
\label{sec:successClassification}

Table~\ref{resultsTable} shows the results of running the two data sets through the
default Weka J48 decision tree algorithm.  Without downsampling, the 
decision tree shows an increase of 4.09\% \textit{correctly classified} instances when adding social data. On the downsampled data, the decision tree created from the technical attributes
had a classification accuracy of 64.10\%, which outperformed the baseline of
guessing all builds passed (50.00\%).
 After including the social data
in the data set, the classifier correctly predicted 76.09\% of the data set,
about a 12\% improvement from only the technical data.  These results show that
both technical and social data can be used to improve predictions of build 
integration success.

It is also worth noticing that in both cases the Kappa statistic increases when
adding social data (100\% and 85\% respectively) which represents a more robust
measure than \textit{correctly classified} instances, since it takes into account the
correctly guessed instances occurring by chance, and we may also notice that the
Mean absolute error decreases, giving stronger confidence to our claims when
considering eventual outcomes. 

\subsection{Build success by core team member}
\label{sec:core}


Based on the decision tree results, we noticed that core team members where
important when classifying build success, so we decided to investigate if core
team members have greater build success rates.
We classified the build success rate for
core team members and non-core team members using the data set without 
downsampling.  We found that core team members
triggered 89.39\% of the builds in the data set and 75.58\% of those builds
passed.  Non-core team members triggered 10.61\% of the builds in the data set
and 64.70\% of the builds passed.  This shows that core team members are more
likely to have a successful build integration than non-core team members.



\subsection{Follower relationship with build success}
\label{sec:followerSuccess}

We also investigated whether developers with followers were more likely to have a
successful build integration, since followers may be seen as vouching for
the developer's abilities.  We specifically investigated if a developer with 
followers was more likely to pass the build in the data set without 
downsampling.  We found that only 0.87\% of the
contributions in the data set was performed by developers with zero 
followers (excluding the developers in which we were not able to extract profile
information for). We found that 99.13\% of the contributions were performed by
developers with at least one or more followers.
Developers with 0 followers had a successful build commit 68.49\%
of the time.  Developers with one or more followers were successful 74.78\% of
the time.

We also investigated developers with a large number of followers (1000 or more
followers).  We found that 6.94\% of the contributions in the data set were
performed by developers with 1000 or more followers.  We found that these
contributions had a 78.61\% success rate.  These results show that developers
with more followers are more likely to have a successful integration build.



\subsection{Discussion}

Overall, we have found that social attributes which proxy experience, such if
the developer is a core contributor for the project, produce a 
noticeable increase in predicting build success.  While the increase was noticeable 
with the full data set, there was a much greater increase in prediction
accuracy after performing downsampling. These results show that social factors
as the ones studied in this paper are significant predictors when
evaluating the predictive power of build commit success.

The build success by core team member results show that core team members are
more likely to have the build integration pass than non-core team members.  This
result provides evidence that non-core team member commit are not thoroughly
scrutinized, although further studied will be required to draw definite
conclusions.

The follower results indicate that developers with more followers are more
likely to pass the build, although followers are not a perfect indicator.
Developers with over 1000 followers had more than a 10\% increase compared to
developers with no followers.  This shows that higher quality developers are more
likely to have a large number of followers.

Overall, this shows that while experience is not a perfect indicator, more
experienced developers are more likely to make high quality project commits.

\section{Threats to validity}

%\paragraph{Internal validity}
 The authors used 
 Weka,\footnote{http://www.cs.waikato.ac.nz/ml/weka/} the well known tool for 
 usage of machine learning algorithms to analyze our datasets. We are aware that, 
 as every human written tool, Weka is prone to contain errors and/or bugs; 
 nonetheless, Weka is a broadly used, open source tool that is subject to high 
 quality standards and a vast user and developer base which makes it more likely 
 to be a high quality code and to have less bugs than lesser used tools.
 
 The scripts we have created to adapt and extract the datasets we needed, and the 
 scripts to obtain the developer social data are publicly available to facilitate 
 the transparency and quality of the process (Will be available after double blind review).%\footnote{https://goo.gl/CqTX44}

%\paragraph{External validity} 
 To tackle the concern that the results of this study might not generalize to 
 other instances outside of the studied cases, we have used a broad and rich 
 dataset~\cite{msr17challenge} in our analysis that comprises 1,359 popular projects, which help to 
 build the notion that since we are building our analysis on top of a vast 
 data set, the threat of this analysis 
 not generalizing to other projects is mitigated.

 Other concerns of the study include the possibility that the social factors
 have changed significantly in the past three years, the collapsing of a
 commit's build success to a single successful build and unsuccessful build
 distorted the results, and the possibility that unsuccessful builds were due to
 incorrect test cases and not low quality contributions.

\section{Conclusion}
In this study we have analyzed how technical and social factors can be used to
predict high quality contributions.  We have found that these factors provide
insight when predicting the success of a build with a technical factors leading
to a 14\% increase in predictive power, and another 12\% increase when
incorporating social factors into the prediction.

We have also found that being a core contributor and having a large follower
base can be an indicator of build integration success.
These results provide evidence that more experienced
developers are more likely to generate a high quality contribution.

% conference papers do not normally have an appendix

% use section* for acknowledgement
\section*{Acknowledgment}

The acknowledgments will be added for the camera ready version.


% trigger a \newpage just before the given reference
% number - used to balance the columns on the last page
% adjust value as needed - may need to be readjusted if
% the document is modified later
%\IEEEtriggeratref{8}
% The "triggered" command can be changed if desired:
%\IEEEtriggercmd{\enlargethispage{-5in}}

% references section

% can use a bibliography generated by BibTeX as a .bbl file
% BibTeX documentation can be easily obtained at:
% http://www.ctan.org/tex-archive/biblio/bibtex/contrib/doc/
% The IEEEtran BibTeX style support page is at:
% http://www.michaelshell.org/tex/ieeetran/bibtex/
%\bibliographystyle{IEEEtran}
% argument is your BibTeX string definitions and bibliography database(s)
%\bibliography{IEEEabrv,../bib/paper}
%
% <OR> manually copy in the resultant .bbl file
% set second argument of \begin to the number of references
% (used to reserve space for the reference number labels box)
%\begin{thebibliography}{1}
%\bibitem{IEEEhowto:kopka}
%H.~Kopka and P.~W. Daly, \emph{A Guide to \LaTeX}, 3rd~ed.\hskip 1em plus
%  0.5em minus 0.4em\relax Harlow, England: Addison-Wesley, 1999.
%\end{thebibliography}

\todo{It is not acceptable to only use the shortened venue name.  Also, bilbiography entries are missing venues again; fix it.}

\bibliographystyle{plain}
\bibliography{MSRChallenge2017}  % sigproc.bib is the name of the Bibliography in this case
% You must have a proper ".bib" file
%  and remember to run:
% latex bibtex latex latex
% to resolve all references



% that's all folks
\end{document}


