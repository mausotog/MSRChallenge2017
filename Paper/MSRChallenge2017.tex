
%% bare_conf.tex
%% V1.3
%% 2007/01/11
%% by Michael Shell
%% See:
%% http://www.michaelshell.org/
%% for current contact information.
%%
%% This is a skeleton file demonstrating the use of IEEEtran.cls
%% (requires IEEEtran.cls version 1.7 or later) with an IEEE conference paper.
%%
%% Support sites:
%% http://www.michaelshell.org/tex/ieeetran/
%% http://www.ctan.org/tex-archive/macros/latex/contrib/IEEEtran/
%% and
%% http://www.ieee.org/

%%*************************************************************************
%% Legal Notice:
%% This code is offered as-is without any warranty either expressed or
%% implied; without even the implied warranty of MERCHANTABILITY or
%% FITNESS FOR A PARTICULAR PURPOSE! 
%% User assumes all risk.
%% In no event shall IEEE or any contributor to this code be liable for
%% any damages or losses, including, but not limited to, incidental,
%% consequential, or any other damages, resulting from the use or misuse
%% of any information contained here.
%%
%% All comments are the opinions of their respective authors and are not
%% necessarily endorsed by the IEEE.
%%
%% This work is distributed under the LaTeX Project Public License (LPPL)
%% ( http://www.latex-project.org/ ) version 1.3, and may be freely used,
%% distributed and modified. A copy of the LPPL, version 1.3, is included
%% in the base LaTeX documentation of all distributions of LaTeX released
%% 2003/12/01 or later.
%% Retain all contribution notices and credits.
%% ** Modified files should be clearly indicated as such, including  **
%% ** renaming them and changing author support contact information. **
%%
%% File list of work: IEEEtran.cls, IEEEtran_HOWTO.pdf, bare_adv.tex,
%%                    bare_conf.tex, bare_jrnl.tex, bare_jrnl_compsoc.tex
%%*************************************************************************

% *** Authors should verify (and, if needed, correct) their LaTeX system  ***
% *** with the testflow diagnostic prior to trusting their LaTeX platform ***
% *** with production work. IEEE's font choices can trigger bugs that do  ***
% *** not appear when using other class files.                            ***
% The testflow support page is at:
% http://www.michaelshell.org/tex/testflow/



% Note that the a4paper option is mainly intended so that authors in
% countries using A4 can easily print to A4 and see how their papers will
% look in print - the typesetting of the document will not typically be
% affected with changes in paper size (but the bottom and side margins will).
% Use the testflow package mentioned above to verify correct handling of
% both paper sizes by the user's LaTeX system.
%
% Also note that the "draftcls" or "draftclsnofoot", not "draft", option
% should be used if it is desired that the figures are to be displayed in
% draft mode.
%
\documentclass[10pt, conference]{IEEEtran}
% Add the compsocconf option for Computer Society conferences.
%
% If IEEEtran.cls has not been installed into the LaTeX system files,
% manually specify the path to it like:
% \documentclass[conference]{../sty/IEEEtran}



\usepackage{booktabs}
\usepackage{verbatim}
\newcommand{\ra}[1]{\renewcommand{\arraystretch}{#1}}

\usepackage[usenames,dvipsnames]{color}
\newcommand{\todo}[1]
  {{\scriptsize \textbf{\color{red} {#1}}}}


% Some very useful LaTeX packages include:
% (uncomment the ones you want to load)


% *** MISC UTILITY PACKAGES ***
%
%\usepackage{ifpdf}
% Heiko Oberdiek's ifpdf.sty is very useful if you need conditional
% compilation based on whether the output is pdf or dvi.
% usage:
% \ifpdf
%   % pdf code
% \else
%   % dvi code
% \fi
% The latest version of ifpdf.sty can be obtained from:
% http://www.ctan.org/tex-archive/macros/latex/contrib/oberdiek/
% Also, note that IEEEtran.cls V1.7 and later provides a builtin
% \ifCLASSINFOpdf conditional that works the same way.
% When switching from latex to pdflatex and vice-versa, the compiler may
% have to be run twice to clear warning/error messages.






% *** CITATION PACKAGES ***
%
\usepackage{cite}
% cite.sty was written by Donald Arseneau
% V1.6 and later of IEEEtran pre-defines the format of the cite.sty package
% \cite{} output to follow that of IEEE. Loading the cite package will
% result in citation numbers being automatically sorted and properly
% "compressed/ranged". e.g., [1], [9], [2], [7], [5], [6] without using
% cite.sty will become [1], [2], [5]--[7], [9] using cite.sty. cite.sty's
% \cite will automatically add leading space, if needed. Use cite.sty's
% noadjust option (cite.sty V3.8 and later) if you want to turn this off.
% cite.sty is already installed on most LaTeX systems. Be sure and use
% version 4.0 (2003-05-27) and later if using hyperref.sty. cite.sty does
% not currently provide for hyperlinked citations.
% The latest version can be obtained at:
% http://www.ctan.org/tex-archive/macros/latex/contrib/cite/
% The documentation is contained in the cite.sty file itself.






% *** GRAPHICS RELATED PACKAGES ***
%
\ifCLASSINFOpdf
  % \usepackage[pdftex]{graphicx}
  % declare the path(s) where your graphic files are
  % \graphicspath{{../pdf/}{../jpeg/}}
  % and their extensions so you won't have to specify these with
  % every instance of \includegraphics
  % \DeclareGraphicsExtensions{.pdf,.jpeg,.png}
\else
  % or other class option (dvipsone, dvipdf, if not using dvips). graphicx
  % will default to the driver specified in the system graphics.cfg if no
  % driver is specified.
  % \usepackage[dvips]{graphicx}
  % declare the path(s) where your graphic files are
  % \graphicspath{{../eps/}}
  % and their extensions so you won't have to specify these with
  % every instance of \includegraphics
  % \DeclareGraphicsExtensions{.eps}
\fi
% graphicx was written by David Carlisle and Sebastian Rahtz. It is
% required if you want graphics, photos, etc. graphicx.sty is already
% installed on most LaTeX systems. The latest version and documentation can
% be obtained at: 
% http://www.ctan.org/tex-archive/macros/latex/required/graphics/
% Another good source of documentation is "Using Imported Graphics in
% LaTeX2e" by Keith Reckdahl which can be found as epslatex.ps or
% epslatex.pdf at: http://www.ctan.org/tex-archive/info/
%
% latex, and pdflatex in dvi mode, support graphics in encapsulated
% postscript (.eps) format. pdflatex in pdf mode supports graphics
% in .pdf, .jpeg, .png and .mps (metapost) formats. Users should ensure
% that all non-photo figures use a vector format (.eps, .pdf, .mps) and
% not a bitmapped formats (.jpeg, .png). IEEE frowns on bitmapped formats
% which can result in "jaggedy"/blurry rendering of lines and letters as
% well as large increases in file sizes.
%
% You can find documentation about the pdfTeX application at:
% http://www.tug.org/applications/pdftex





% *** MATH PACKAGES ***
%
%\usepackage[cmex10]{amsmath}
% A popular package from the American Mathematical Society that provides
% many useful and powerful commands for dealing with mathematics. If using
% it, be sure to load this package with the cmex10 option to ensure that
% only type 1 fonts will utilized at all point sizes. Without this option,
% it is possible that some math symbols, particularly those within
% footnotes, will be rendered in bitmap form which will result in a
% document that can not be IEEE Xplore compliant!
%
% Also, note that the amsmath package sets \interdisplaylinepenalty to 10000
% thus preventing page breaks from occurring within multiline equations. Use:
%\interdisplaylinepenalty=2500
% after loading amsmath to restore such page breaks as IEEEtran.cls normally
% does. amsmath.sty is already installed on most LaTeX systems. The latest
% version and documentation can be obtained at:
% http://www.ctan.org/tex-archive/macros/latex/required/amslatex/math/





% *** SPECIALIZED LIST PACKAGES ***
%
%\usepackage{algorithmic}
% algorithmic.sty was written by Peter Williams and Rogerio Brito.
% This package provides an algorithmic environment fo describing algorithms.
% You can use the algorithmic environment in-text or within a figure
% environment to provide for a floating algorithm. Do NOT use the algorithm
% floating environment provided by algorithm.sty (by the same authors) or
% algorithm2e.sty (by Christophe Fiorio) as IEEE does not use dedicated
% algorithm float types and packages that provide these will not provide
% correct IEEE style captions. The latest version and documentation of
% algorithmic.sty can be obtained at:
% http://www.ctan.org/tex-archive/macros/latex/contrib/algorithms/
% There is also a support site at:
% http://algorithms.berlios.de/index.html
% Also of interest may be the (relatively newer and more customizable)
% algorithmicx.sty package by Szasz Janos:
% http://www.ctan.org/tex-archive/macros/latex/contrib/algorithmicx/




% *** ALIGNMENT PACKAGES ***
%
%\usepackage{array}
% Frank Mittelbach's and David Carlisle's array.sty patches and improves
% the standard LaTeX2e array and tabular environments to provide better
% appearance and additional user controls. As the default LaTeX2e table
% generation code is lacking to the point of almost being broken with
% respect to the quality of the end results, all users are strongly
% advised to use an enhanced (at the very least that provided by array.sty)
% set of table tools. array.sty is already installed on most systems. The
% latest version and documentation can be obtained at:
% http://www.ctan.org/tex-archive/macros/latex/required/tools/


%\usepackage{mdwmath}
%\usepackage{mdwtab}
% Also highly recommended is Mark Wooding's extremely powerful MDW tools,
% especially mdwmath.sty and mdwtab.sty which are used to format equations
% and tables, respectively. The MDWtools set is already installed on most
% LaTeX systems. The lastest version and documentation is available at:
% http://www.ctan.org/tex-archive/macros/latex/contrib/mdwtools/


% IEEEtran contains the IEEEeqnarray family of commands that can be used to
% generate multiline equations as well as matrices, tables, etc., of high
% quality.


%\usepackage{eqparbox}
% Also of notable interest is Scott Pakin's eqparbox package for creating
% (automatically sized) equal width boxes - aka "natural width parboxes".
% Available at:
% http://www.ctan.org/tex-archive/macros/latex/contrib/eqparbox/





% *** SUBFIGURE PACKAGES ***
%\usepackage[tight,footnotesize]{subfigure}
% subfigure.sty was written by Steven Douglas Cochran. This package makes it
% easy to put subfigures in your figures. e.g., "Figure 1a and 1b". For IEEE
% work, it is a good idea to load it with the tight package option to reduce
% the amount of white space around the subfigures. subfigure.sty is already
% installed on most LaTeX systems. The latest version and documentation can
% be obtained at:
% http://www.ctan.org/tex-archive/obsolete/macros/latex/contrib/subfigure/
% subfigure.sty has been superceeded by subfig.sty.



%\usepackage[caption=false]{caption}
%\usepackage[font=footnotesize]{subfig}
% subfig.sty, also written by Steven Douglas Cochran, is the modern
% replacement for subfigure.sty. However, subfig.sty requires and
% automatically loads Axel Sommerfeldt's caption.sty which will override
% IEEEtran.cls handling of captions and this will result in nonIEEE style
% figure/table captions. To prevent this problem, be sure and preload
% caption.sty with its "caption=false" package option. This is will preserve
% IEEEtran.cls handing of captions. Version 1.3 (2005/06/28) and later 
% (recommended due to many improvements over 1.2) of subfig.sty supports
% the caption=false option directly:
%\usepackage[caption=false,font=footnotesize]{subfig}
%
% The latest version and documentation can be obtained at:
% http://www.ctan.org/tex-archive/macros/latex/contrib/subfig/
% The latest version and documentation of caption.sty can be obtained at:
% http://www.ctan.org/tex-archive/macros/latex/contrib/caption/




% *** FLOAT PACKAGES ***
%
%\usepackage{fixltx2e}
% fixltx2e, the successor to the earlier fix2col.sty, was written by
% Frank Mittelbach and David Carlisle. This package corrects a few problems
% in the LaTeX2e kernel, the most notable of which is that in current
% LaTeX2e releases, the ordering of single and double column floats is not
% guaranteed to be preserved. Thus, an unpatched LaTeX2e can allow a
% single column figure to be placed prior to an earlier double column
% figure. The latest version and documentation can be found at:
% http://www.ctan.org/tex-archive/macros/latex/base/



%\usepackage{stfloats}
% stfloats.sty was written by Sigitas Tolusis. This package gives LaTeX2e
% the ability to do double column floats at the bottom of the page as well
% as the top. (e.g., "\begin{figure*}[!b]" is not normally possible in
% LaTeX2e). It also provides a command:
%\fnbelowfloat
% to enable the placement of footnotes below bottom floats (the standard
% LaTeX2e kernel puts them above bottom floats). This is an invasive package
% which rewrites many portions of the LaTeX2e float routines. It may not work
% with other packages that modify the LaTeX2e float routines. The latest
% version and documentation can be obtained at:
% http://www.ctan.org/tex-archive/macros/latex/contrib/sttools/
% Documentation is contained in the stfloats.sty comments as well as in the
% presfull.pdf file. Do not use the stfloats baselinefloat ability as IEEE
% does not allow \baselineskip to stretch. Authors submitting work to the
% IEEE should note that IEEE rarely uses double column equations and
% that authors should try to avoid such use. Do not be tempted to use the
% cuted.sty or midfloat.sty packages (also by Sigitas Tolusis) as IEEE does
% not format its papers in such ways.





% *** PDF, URL AND HYPERLINK PACKAGES ***
%
%\usepackage{url}
% url.sty was written by Donald Arseneau. It provides better support for
% handling and breaking URLs. url.sty is already installed on most LaTeX
% systems. The latest version can be obtained at:
% http://www.ctan.org/tex-archive/macros/latex/contrib/misc/
% Read the url.sty source comments for usage information. Basically,
% \url{my_url_here}.





% *** Do not adjust lengths that control margins, column widths, etc. ***
% *** Do not use packages that alter fonts (such as pslatex).         ***
% There should be no need to do such things with IEEEtran.cls V1.6 and later.
% (Unless specifically asked to do so by the journal or conference you plan
% to submit to, of course. )


% correct bad hyphenation here
\hyphenation{op-tical net-works semi-conduc-tor}


\begin{document}
%
% paper title
% can use linebreaks \\ within to get better formatting as desired

\title{Analyzing the impact of social attributes on the success of commit integration}
%Or some better name, Zack can probably think of a better one, he is good with names


% author names and affiliations
% use a multiple column layout for up to two different
% affiliations

\author{\IEEEauthorblockN{Authors hidden for the purposes of double blind review}
\IEEEauthorblockA{\\
\\
\\}

%\author{\IEEEauthorblockN{Authors Name/s per 1st Affiliation (Author)}
%\IEEEauthorblockA{line 1 (of Affiliation): dept. name of organization\\
%line 2: name of organization, acronyms acceptable\\
%line 3: City, Country\\
%line 4: Email: name@xyz.com}
%\and
%\IEEEauthorblockN{Authors Name/s per 2nd Affiliation (Author)}
%\IEEEauthorblockA{line 1 (of Affiliation): dept. name of organization\\
%line 2: name of organization, acronyms acceptable\\
%line 3: City, Country\\
%line 4: Email: name@xyz.com}
}

% conference papers do not typically use \thanks and this command
% is locked out in conference mode. If really needed, such as for
% the acknowledgment of grants, issue a \IEEEoverridecommandlockouts
% after \documentclass

% for over three affiliations, or if they all won't fit within the width
% of the page, use this alternative format:
% 
%\author{\IEEEauthorblockN{Michael Shell\IEEEauthorrefmark{1},
%Homer Simpson\IEEEauthorrefmark{2},
%James Kirk\IEEEauthorrefmark{3}, 
%Montgomery Scott\IEEEauthorrefmark{3} and
%Eldon Tyrell\IEEEauthorrefmark{4}}
%\IEEEauthorblockA{\IEEEauthorrefmark{1}School of Electrical and Computer Engineering\\
%Georgia Institute of Technology,
%Atlanta, Georgia 30332--0250\\ Email: see http://www.michaelshell.org/contact.html}
%\IEEEauthorblockA{\IEEEauthorrefmark{2}Twentieth Century Fox, Springfield, USA\\
%Email: homer@thesimpsons.com}
%\IEEEauthorblockA{\IEEEauthorrefmark{3}Starfleet Academy, San Francisco, California 96678-2391\\
%Telephone: (800) 555--1212, Fax: (888) 555--1212}
%\IEEEauthorblockA{\IEEEauthorrefmark{4}Tyrell Inc., 123 Replicant Street, Los Angeles, California 90210--4321}}




% use for special paper notices
%\IEEEspecialpapernotice{(Invited Paper)}




% make the title area
\maketitle


\begin{abstract}
One important aspect of developing open source software is deciding which
commits to include in the main branch of a project.  
Recent research has started to explore how the open source community manages
this question and found that both technical and social factors are factors in 
the success of project contributions on GitHub.  
This investigation defined a successful project contribution as a project
contribution that gets
accepted to the project.  However, with the continuous integration, we can
explore this question a step further.  We can better estimate the success of a
contribution by investigating if the contribution passed the project's
test cases.  We performed an initial investigation into the technical and
social factors involved in a commit that successfully passes the integration
tests. Specifically, we investigated if social factors (such as being a
core member of the development team, having a large number of followers,
or contributing a large number of commits) improve predictions of build
success.  We have found that while social factors cause an noticeable increase
in predictive power (12\%), further investigation into these factors are
required.
\end{abstract}

\begin{IEEEkeywords}
Social Attributes; Travis CI; Github; Social Networks; Predicting Integration 
Success; Social Coding
\end{IEEEkeywords}


% For peer review papers, you can put extra information on the cover
% page as needed:
% \ifCLASSOPTIONpeerreview
% \begin{center} \bfseries EDICS Category: 3-BBND \end{center}
% \fi
%
% For peerreview papers, this IEEEtran command inserts a page break and
% creates the second title. It will be ignored for other modes.
\IEEEpeerreviewmaketitle



\section{Introduction}
For managers of an open source project, it is important to ensure that the
project consists of high quality contributions.  The decision to include code in
a project has been described as a meritocracy, where only the code is a factor
in the decision~\cite{Scacchi07}.  However, to better understand successful 
project contributions, we must analyze factors which are common to all commits.  
  


Over the years, the open source
community has developed various techniques to vet project contributions, such as pull
requests which must be approved by a core team member~\cite{gousios14}. One such
technique is continuous integration. Continuous integration automatically checks
the contribution against the project's test cases, to notify the development
team if the change caused a test case failure~\cite{Meyer14}.  When a change
passes one of these vetting techniques, it shows that the project developers
have confidence in the changes.

In past studies, both technical factors and social factors have been found to be
important when when predicting pull request success~\cite{tsay14icse,tsay14fse}.
Yet, these factors may not be as important during continuous integration.
Specifically, we want to investigate if social factors, which can be used a
proxy for experience, are important when predicting build success when grouped
with technical data, used as a proxy for commit complexity.

The knowledge obtained by 
analyzing how much impact social characteristics have on the success of a commit 
integration will help developers further understand what social attributes are 
better proxies for experience and coding excellence; and it will also help code 
integrators and core team members to know how to prioritize pull requests coming 
from different developers, which of them may have a higher probability of 
integration success based not only on the technical difficulty of the 
integration, but also on social factors that determine the expertise of the 
developer in their integration practices.

\textbf{Related work:}
Similar studies have shown the relationship between different attributes and 
different measures of success: pull request acceptance~\cite{gousios14, 
gousios15,tsay14icse,tsay14fse}, pull request evaluation latency~\cite{Yu15}.
Tsay et al~\cite{tsay14icse} performed a very relevant study where they 
determine that the social connection between the developer who is asking the 
pull request and the integrators is a key factor torwards acceptance of a pull 
request. Unlike this previous study, in our study we look at a broader aspect of 
social aspects and how this attributes towards the success in Travis CI jobs. 
Gousios et al~\cite{gousios14} study the relationship between the area of the 
source code being modified by the developer requesting the changes to be pulled 
and the acceptance rate of these changes is highly correlated. Also existence of 
tests and overall code quality~\cite{gousios15} are attributes that have a 
strong relationship with the acceptance rate of a pull request. Bettenburgh et 
al~\cite{bettenburgh10} discuss the impact of social interactions with 
post-release defects, Vasilescu et al~\cite{vasilescu14} explored how different 
project characteristics (programming language, project age, etc.) affect commit 
success when integrated to the automatic build process. Yu et al~\cite{yu16} 
analyze the impact of CI usage in Github and its correlation with software 
quality. Unlike the works previously listed, our study focuses on independent social 
attributes and how these relate to the success of the integration of a commit 
pushed by the developer being described by these social attributes.

Our study shows that social factors are an important aspect of build success but
these factors alone do not fully predict if a build will succeed.  We also show
that core team members are more likely to achieve build success than non-core
team members and having followers increases the likelihood passing the
continuous integration build.

\section{Research Design}
We present the approach taken to guide this study by
following the Goal-Question-Metric (GQM) methodology descibed by Basili et 
al~\cite{Basili84}.

\subsection{Goal}
The goal of this study is to have a deeper understanding of the relationship 
between the several different social properties that that estimate the
developer's experience, and the likelihood of a developer's commit
to be considered a successfully integrated in the project. We define successful
integration as passing the continuous integration build tests in the project.

\subsection{Question}

Previous studies have identified a relationship between technical and social 
factors and pull request acceptance. We are interested in understanding if
social factors, that proxy experience, can be used to predict
a different success measure: continuous integration success.

Therefore our main research question is presented as follows:
\newline \newline
\textbf{Do social attributes, gathered 
from the Github profiles of developers,
allow for more accurate predictions than only using technical data?}
\newline \newline

Depending on the results of this question, we can further explore which
attributes are most important for predicting build integration success.

\begin{comment}
we have created a dataset with technical attributes to be able to analyze the predictiveness of these attributes and its relationship with commit integration success. This way, we are able to create a model
We are afterwards interested in knowing if adding social attributes to this 
predictability model will increase the probability of the model asserting 
correctly future instances by knowing not only technical data about the commit 
being pushed but also social attributes of the developer pushing the changes.
\end{comment}


\subsection{Metric}
We created a linear regression model based on two different sets of data. Merely 
technical data related to technical aspects of the complexity of the commit, and 
a superset of this before-mentioned data and social data gathered from the 
developer who pushed the commit. We evaluate the predictive power of these two 
models with 10 fold cross validation and we analyze which of them is able to 
predict with higher accuracy the success of unseen commits.

\section{Building Dataset}
\label{approach}
To investigate our research questions, we built a dataset based on the 
attributes that consisted of both technical and social factors.  We collected
the technical factors from the provided data set~\cite{msr17challenge}.
We extracted only the attributes in the data set that proxy the commit's
complexity and the build status.
\begin{itemize}
%\item tr\_original\_commit
%\item gh\_build\_started\_at 
\item git\_diff\_src\_churn 
\item gh\_diff\_files\_added 
\item gh\_diff\_files\_deleted 
\item gh\_diff\_files\_modified 
\item gh\_diff\_tests\_added 
\item gh\_diff\_tests\_deleted 
\item gh\_diff\_src\_files 
\item gh\_sloc 
\item tr\_status 
\end{itemize}

Once we had these attributes, we removed repeated rows \todo{Why did we have to
remove repeated rows?}

To collect the social data, we created a web scrapper to extract the social data
from developer's profiles.
We used the commit hash and the project name in the original data
set~\cite{msr17challenge} to locate the web page of the commit using the syntax
below. 
\newline \newline
\textit{https://github.com/\textbf{NameOfProject}/commit/\textbf{CommitHash}}
\newline \newline
Then, we scraped the commit author's
\textit{username} from the page. If the scrapper was able to find username, the
scraper went to the user's profile information using the syntax below.
\newline \newline
\textit{https://github.com/\textbf{UserName}}
\newline \newline
The scrapper then extracted the following items from the user's profile.
\begin{itemize}
\item Repositories
\item Stars
\item Followers
\item Following
\item Contributions 
\end{itemize}

Repositories represents the repositories the developer has access to. This 
includes where the developer is the repository owner,where the developer is a 
collaborator, and repositories that the developer can access through an 
organization membership.\footnote{https://developer.github.com/v3/repos/} Stars 
represents a count of the projects the developer has stared. Starring  a project
means to showing approval of the repository and creating a bookmark for later 
access.\footnote{https://help.github.com/articles/about-stars/} Followers and 
Following detail the number of developers that are following and being followed 
by the developer in question. Finally, the last attribute studied is the number 
of contributions made by the developer in the last year. 

We also added one item from the original data set to the social factors:
gh\_by\_core\_team\_member.
This factor indicates if the developer making a contribution is a core member 
of that project.

We used this data to create two data sets.  One with only the technical
factors and the build status, and another with both the technical and social
factors along with the build status.

The dataset used in this study~\cite{msr17challenge} is very diverse 
and consists of commits that date from far back, and the social data we are 
gathering is recent data, gathered in January 2017. The researchers of this 
project acknowledge this discrepancy and in an effort to mitigate the threat of
using future result to predict previous events,
we removed the commits in the dataset that were not performed within the last
three years of the data set's release (2014,2015,2016).  While this is not
perfect, we believe that these numbers are unlikely to have changed
significantly in the last three years.

After building the data set, we converted the various Travis build statuses to
successful (passed in the data set) or not successful (errored, failed, canceled
and started in the data set).  This still left a big discrepancy in the two
categories, with about a 3:1 successful to not successful ratio. To avoid having
this discrepancy alter our results, we created a separate data set where the
successful builds were randomly selected until the number of successful builds
equaled the number of not successful builds (downsampled\todo{citation needed}).  We used both the
downsampled and non-downsampled data sets in our results.


\section{Evaluation}

We evaluated the predictive power of the social factors using a decision
tree~\cite{Quinlan86}.
Decision trees are a predictive model which uses binary decisions on the
attributes to predict a final class.  The predicted class is represented as the
leaves, while the attribute decisions used in the prediction cause the
different branches.
In this study, we used the Weka~\cite{Weka} J48~\cite{Quinlan1993} decision tree
with default settings. The decision trees were evaluated using ten fold cross 
validation with the data in each data set.

\section{Results}

\subsection{Success Classification}
\begin{table*}[ht]
\centering
\begin{tabular}{l||r|r||r|r}
  & \multicolumn{2}{|c||}{Original number of instances}  & \multicolumn{2}{|c}{Downsampled} \\
\hline
  & Technical & Technical and Social & Technical & Technical and Social \\
\hline
\textbf{Correctly Classified Instances}&  77.01\% & 81.10\%  & 64.10\% & 76.09\%\\

\textbf{Incorrectly Classified Instances}&  22.99\% & 18.90\% & 35.90\% & 23.91\%\\

\textbf{Kappa statistic}& 0.21 &  0.42 & 0.28  & 0.52 \\

\textbf{Mean absolute error}& 0.34 &  0.28 & 0.43 & 0.32\\

\textbf{Relative absolute error}& 88.64\% & 72.42\%  & 85.35\% & 63.07\%\\
\hline
\textbf{Increase when adding social attributes}& \multicolumn{2}{|c||}{4.09\%}  & \multicolumn{2}{|c}{11.99\%}\\

\hline
\end{tabular}
\\
\center
  \caption{ Second and third columns: Comparison of performance when running 10 fold cross validation of two decision trees. The first column after the first double bar uses only technical attributes to predict the Travis Job status, the second column augments the dataset with social attributes. Fourth and fifth columns: The same intuition as the previous two, with the difference that the dataset was downsampled to provide the same number of successful and unsuccessful instances }
  \label{resultsTable}
\end{table*} 

 
In the original data set, 74.43\% of the builds were successful, the rest were not
successful.  This means that a classifier which guesses successful every time
would produce a classification rate of 74.43\% if ran on the full data set. 

Table~\ref{resultsTable} shows the results of running the two data sets through the
default Weka J48 decision tree algorithm.The decision tree built from
only the technical indicators has a correct classification rate of 77.01\% which
means the technical attributes only increased the prediction accuracy by about 3\%.
Adding the social data increased the classifier's accuracy from 77.01\% to
81.10\%, about a 4\% increase.  

On the downsampled data, the decision tree created from the technical attributes
had a classification accuracy of 64.10\%, when predicting a successful build in
all instances would produce 50.00\% accuracy.  After including the social data
in the data set, the classifier correctly predicted 76.09\% of the data set,
about a 12\% improvement from only the technical data.


\subsection{Build success by core team member}

For the build success investigation, we classified the build success rate for
core team members and non-core team members.  We found that core team members
triggered of 89.39\% of the builds in the data set and 75.58\% of those builds
passed.  Non-core team members triggered 10.61\% of the builds in the data set
and 64.70\% of the builds passed.


\subsection{Follower effect on build success}

We also investigated if developers with followers were more likely to have a
successful build integration, since have followers may be seen as vouching for
the developer's abilities.  We specifically investigated if a developer with 
followers was more likely to pass the build.  We found that only 0.87\% of the
contributions in the data set was performed by developers with zero 
followers (excluding the developers in which we were not able to extract profile
information for). We found that 99.13\% of the contributions were performed by
developers with at least one or more followers.
We found that developers with 0 followers had a successful build commit 68.49\%
of the time.  Developers with one or more followers were successful 74.78\% of
the time.

We also investigated developers with a large number of followers (1000 or more
followers).  We found that 6.94\% of the contributions in the data set were
performed by developers with 1000 or more followers.  We found that these
contributions had a 78.61\% success rate.



\section{Discussion}
We have found that social attributes which proxy experience produce a small but
noticeable increase in predicting build success.  While the increase was very
small with the full data set, there was a greater increase in prediction
accuracy after performing downsampling.  These results show that experience is
important to predicting build commit success but a strong predictor.

The build success by core team member results show that core team members are
more likely to have the build integration pass than non-core team members.  This
result provides evidence that non-core team member commit are not throughly
scrutinized, although further studied will be required to draw definite
conclusions.

The follower results indicate that developers with more followers are more
likely to pass the build, although followers are not a perfect indicator.
Developers with over 1000 followers had more than a 10\% increase compared to
developers with no followers.  This shows that higher quality developers are more
likely to have a large number of followers.

% An example of a floating figure using the graphicx package.
% Note that \label must occur AFTER (or within) \caption.
% For figures, \caption should occur after the \includegraphics.
% Note that IEEEtran v1.7 and later has special internal code that
% is designed to preserve the operation of \label within \caption
% even when the captionsoff option is in effect. However, because
% of issues like this, it may be the safest practice to put all your
% \label just after \caption rather than within \caption{}.
%
% Reminder: the "draftcls" or "draftclsnofoot", not "draft", class
% option should be used if it is desired that the figures are to be
% displayed while in draft mode.
%
%\begin{figure}[!t]
%\centering
%\includegraphics[width=2.5in]{myfigure}
% where an .eps filename suffix will be assumed under latex, 
% and a .pdf suffix will be assumed for pdflatex; or what has been declared
% via \DeclareGraphicsExtensions.
%\caption{Simulation Results}
%\label{fig_sim}
%\end{figure}

% Note that IEEE typically puts floats only at the top, even when this
% results in a large percentage of a column being occupied by floats.


% An example of a double column floating figure using two subfigures.
% (The subfig.sty package must be loaded for this to work.)
% The subfigure \label commands are set within each subfloat command, the
% \label for the overall figure must come after \caption.
% \hfil must be used as a separator to get equal spacing.
% The subfigure.sty package works much the same way, except \subfigure is
% used instead of \subfloat.
%
%\begin{figure*}[!t]
%\centerline{\subfloat[Case I]\includegraphics[width=2.5in]{subfigcase1}%
%\label{fig_first_case}}
%\hfil
%\subfloat[Case II]{\includegraphics[width=2.5in]{subfigcase2}%
%\label{fig_second_case}}}
%\caption{Simulation results}
%\label{fig_sim}
%\end{figure*}
%
% Note that often IEEE papers with subfigures do not employ subfigure
% captions (using the optional argument to \subfloat), but instead will
% reference/describe all of them (a), (b), etc., within the main caption.


% An example of a floating table. Note that, for IEEE style tables, the 
% \caption command should come BEFORE the table. Table text will default to
% \footnotesize as IEEE normally uses this smaller font for tables.
% The \label must come after \caption as always.
%
%\begin{table}[!t]
%% increase table row spacing, adjust to taste
%\renewcommand{\arraystretch}{1.3}
% if using array.sty, it might be a good idea to tweak the value of
% \extrarowheight as needed to properly center the text within the cells
%\caption{An Example of a Table}
%\label{table_example}
%\centering
%% Some packages, such as MDW tools, offer better commands for making tables
%% than the plain LaTeX2e tabular which is used here.
%\begin{tabular}{|c||c|}
%\hline
%One & Two\\
%\hline
%Three & Four\\
%\hline
%\end{tabular}
%\end{table}


% Note that IEEE does not put floats in the very first column - or typically
% anywhere on the first page for that matter. Also, in-text middle ("here")
% positioning is not used. Most IEEE journals/conferences use top floats
% exclusively. Note that, LaTeX2e, unlike IEEE journals/conferences, places
% footnotes above bottom floats. This can be corrected via the \fnbelowfloat
% command of the stfloats package.


\section{Threats to validity}
This study contains various threats to validity.

\noindent\textbf{Internal validity:}\\
One source of internal validity concern is the time mismatch between the
technical and social data.  We restricted the commits to only the commits in the
last three years of the data set to reduce this concern.
There is also a chance that scripts we used to create the data sets contain an
error.  For full transparency, we've released the scripts used to collect the
data set at (This will be added after the anonymous review).

%footnote{https://goo.gl/CqTX44}

\noindent\textbf{External validity:} \\
While we have used a data set with high quality standards~\cite{msr17challenge},
these results may not generalized to non-open source projects.

\noindent\textbf{Construct validity:}\\
While we believe the build status is a good metric for contribution success,
there is the chance that a problem in the project's test cases, and not the
contribution itself, led to the build failure.  There are also other possible
technical and social factors which were not explored in this study.  Future
research will be required to determine if those factors are important.

\section{Conclusion}
In this study we have analyzed the ability to use social factors to predict
build integration success.
Our results show that adding social data, which proxies experience, provides a
small increases in prediction accuracy.  This leads us to conclude that while
experience is important, it is not the only factor in build success.
We have also found that non-core contributors cause not successful builds more
than core contributors.  This leads us to believe that non-core contributions
are not inspected as closely as they should be.  We have also found that having
a follower, and especially a large follower base, can be an indicator of build
integration success. 

% conference papers do not normally have an appendix

% use section* for acknowledgement
\section*{Acknowledgment}

The acknowledgments will be added for the camera ready version.


% trigger a \newpage just before the given reference
% number - used to balance the columns on the last page
% adjust value as needed - may need to be readjusted if
% the document is modified later
%\IEEEtriggeratref{8}
% The "triggered" command can be changed if desired:
%\IEEEtriggercmd{\enlargethispage{-5in}}

% references section

% can use a bibliography generated by BibTeX as a .bbl file
% BibTeX documentation can be easily obtained at:
% http://www.ctan.org/tex-archive/biblio/bibtex/contrib/doc/
% The IEEEtran BibTeX style support page is at:
% http://www.michaelshell.org/tex/ieeetran/bibtex/
%\bibliographystyle{IEEEtran}
% argument is your BibTeX string definitions and bibliography database(s)
%\bibliography{IEEEabrv,../bib/paper}
%
% <OR> manually copy in the resultant .bbl file
% set second argument of \begin to the number of references
% (used to reserve space for the reference number labels box)
%\begin{thebibliography}{1}
%\bibitem{IEEEhowto:kopka}
%H.~Kopka and P.~W. Daly, \emph{A Guide to \LaTeX}, 3rd~ed.\hskip 1em plus
%  0.5em minus 0.4em\relax Harlow, England: Addison-Wesley, 1999.
%\end{thebibliography}


\bibliographystyle{plain}
\bibliography{MSRChallenge2017}  % sigproc.bib is the name of the Bibliography in this case
% You must have a proper ".bib" file
%  and remember to run:
% latex bibtex latex latex
% to resolve all references



% that's all folks
\end{document}


